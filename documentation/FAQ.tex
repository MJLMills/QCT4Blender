\chapter{FAQ}

\newthought{As very few} people have been supplied a copy of \programName{} at this stage, correspondingly few questions have been asked, and none frequently. Therefore, please consider this section 'Fearfully Anticipated Questions' until such time as it becomes clear what people are actually concerned about after R'ing the FM. If your question is missing or the answer is not clear please email the author at mjohnmills@gmail.com.

1) "How do you..." 

Let me stop you already. Have you read the manual? Carefully? If so then please continue. Otherwise go and read the manual. Carefully.

2) "Can I: search for critical points; trace gradient paths; triangulate surfaces of a scalar function etc. with \programName{}?

No. \programName{} does not perform any topological analysis. All such analysis is left to dedicated programs such as MORPHY and AIMAll.

3) Can I improve the resolution of topological elements with \programName{}?

Blender provides powerful algorithms for smoothly interpolating lines between points and for adding more points on an already defined surface. From experience, these algorithms work nicely to improve the resolution of gradient paths and interatomic/capping surfaces. However, the points added by these methods are not rigorously determined from the electron density and so care must be exercised in their use to avoid the introduction of artifacts.

4) How can I change the default appearance or add unsupported elements?

The colours and critical point radii are defined in .json data files provided with the program. These are read by the script on each invocation and so the simplest way to change defaults for these properties is to edit the .json. These files are written in Python. Radii are specified in the dictionary 'elementRadii' as pairs of element name and value. Colours are specified in the dictionary 'elementColors' as pairs of element name and RGB values.

Further properties will be settable in a future general configuration file.

5) How can I 