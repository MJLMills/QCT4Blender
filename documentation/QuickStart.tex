\documentclass{tufte-handout}

\newcommand{\blenderVersion}{2.76b}
\newcommand{\programName}{RhoRix}
\newcommand{\fullName}{RhoRix - A Program for Drawing of Quantum Chemical Topologies by M J L Mills}
\newcommand{\programWebsite}{www.mjlmills.com/rhorix}
\newcommand{\programCitation}{Mills \& Popelier, in Preparation}
\newcommand{\addonPath}{User Preferences $\rightarrow$ Add-Ons}
\newcommand{\operatorPath}{File $\rightarrow$ Import $\rightarrow$ Quantum Chemical Topology (.top)}
\newcommand{\navLink}{www.blender.org/manual/editors/3dview/navigate/camera\_view.html}
\newcommand{\blenderSite}{www.blender.org}
\newcommand{\manualLink}{www.blender.org/manual/}
\newcommand{\enterCamera}{keypad $0$}
\newcommand{\flyMode}{Shift-F}
\newcommand{\renderKey}{F12}
\newcommand{\leaveRender}{Esc}

\usepackage{graphicx} % allow embedded images
 \setkeys{Gin}{width=\linewidth,totalheight=\textheight,keepaspectratio}
 \graphicspath{{graphics/}} % set of paths to search for images
\usepackage{amsmath}  % extended mathematics
\usepackage{booktabs} % book-quality tables
\usepackage{units}    % non-stacked fractions and better unit spacing
\usepackage{multicol} % multiple column layout facilities
\usepackage{lipsum}   % filler text
\usepackage{fancyvrb} % extended verbatim environments
 \fvset{fontsize=\normalsize}% default font size for fancy-verbatim environments

% Standardize command font styles and environments
\newcommand{\doccmd}[1]{\texttt{\textbackslash#1}}% command name -- adds backslash automatically
\newcommand{\docopt}[1]{\ensuremath{\langle}\textrm{\textit{#1}}\ensuremath{\rangle}}% optional command argument
\newcommand{\docarg}[1]{\textrm{\textit{#1}}}% (required) command argument
\newcommand{\docenv}[1]{\textsf{#1}}% environment name
\newcommand{\docpkg}[1]{\texttt{#1}}% package name
\newcommand{\doccls}[1]{\texttt{#1}}% document class name
\newcommand{\docclsopt}[1]{\texttt{#1}}% document class option name
\newenvironment{docspec}{\begin{quote}\noindent}{\end{quote}}% command specification environment


\title{\programName{} - Quick Start Guide}
\author{Matthew J L Mills}

\begin{document}

\maketitle

% provide already setup default environment for rendering as a blender settings file

\begin{abstract}
\programName{} is a program that allows the user to model (in 3D space) and render (in 2D flatland) a representation of the quantum chemical topology (QCT) of a chemical system with the powerful Blender\sidenote{\blenderSite{}} tool. 
This 'Quick Start' document is intended for those users currently without interest in how the program works who simply want to use it. These instructions provide the quickest path from QCT calculation to rendered image, explaining how to get straight into using \programName{} with Blender to generate a standard-appearance rendered image from a QCT data file.
\par{}
Those users requiring more detail should consult the \programName{} and Blender Manuals\sidenote{\manualLink{}}, and all users are urged to look therein for answers not present in this guide.

\end{abstract}

\section{Obtaining the Code}
\programName{} can currently only be obtained directly from the author\sidenote{mjohnmills@gmail.com} by sending a request message, including your name and place of work. 
The supplied archive contains the Blender Add-On Python script, its data files, the filetype conversion Perl scripts and documentation. The script and data files must be placed in an appropriate directory (of your choosing) prior to following the installation instructions below.

\section{Installing the Add-On}
The currently supported version (\blenderVersion{}) of Blender must be installed before proceeding with the \programName{} installation. Installing Blender is not covered here; please consult the Blender website for instructions for your OS\sidenote{Windows, Mac OSX, GNU/Linux and FreeBSD are supported.}. 
Once you have a working version of Blender in place, the script can be installed within it.
Users with existing Blender installations and setups can directly install the script following the instructions below.
New users are encouraged to first load the provided Blender settings file, which will arrange the internal windows appropriately for the given task.
These settings can then be saved as defaults for future projects if needed.
\par{}
Installation of the script can be carried out from inside Blender.
Open Blender, navigate to User Preferences and choose the 'Add-Ons' tab\sidenote{\addonPath{}}.
Click "Install Add-On" and use the file viewer to locate the script on your machine and select it.
This will copy the script to you personal Add-Ons directory, and it will now appear in the list of available Add-Ons.
To activate it you must tick the checkbox for that entry. 
You can then save the Blender configuration for all future documents, or alternatively tick the Add-On in each document you use for QCT drawing.
The script, when the checkbox is selected, will add an operator named 'Import Topology' to Blender as a menu item\sidenote{\operatorPath{}}.
Clicking this button will open a file viewer allowing navigation to a .top file.

\section{Converting QCT Output Files}
In order to model and render a topology, the topology must first be known. 
Therefore a QCT calculation is the first step in producing an image. 
Instructions on producing the data necessary to render an image of the topology should be found in the documentation of your particular QCT analysis program.
\programName{} currently supports MORPHY and AIMAll output files (and by extension supports any \textit{ab initio} code that produces an AIM wavefunction), which must be converted to the .top filetype before they can be read into Blender. 
Several Perl scripts are provided for this purpose, named for the filetype that they convert from.

\section{Reading Topology Files}
After successful conversion, the .top file can be read into Blender via \programName{} by clicking the appropriate menu item\sidenote{\operatorPath{}} and navigating to the file's location.
The appearance of the topology will follow the default template, i.e. standard colors and sphere radii will be used as listed in the Manual.
Following successful reading of a .top file, a 3D model of the topology will appear. To the right of the 3D model, n the outliner window, a folder structure appears which breaks the topology down into sets of elements for easy navigation and selection.

\section{Placing the Camera \& Rendering an Image}
The camera must be positioned such that the appropriate part of the topology will be rendered onto the 2D image plane.
Rendering can be imagined as the process of taking a picture of the topology with the virtual camera. 
Thus, the camera settings all affect the outcome. 
The user is referred to the appropriate sections of the Blender documentation for descriptions of these settings.
The minimal demand on the user is typically to move the camera to get the desired part of the system in the final 
render.
\programName{} automatically positions the camera such that it is outside the system and points at the origin.This is intended to provide an adequate starting point for placing the camera.
\par{}
There are 2 important 3D viewports for the quickstart user. The program will show the 3D view window once the 
system is read in.
The purpose of this view is to allow the manipulation of objects and materials, and provide a pre-rendered image 
of the scene.
The second viewport is the Camera View. The camera view shows you the orientation of the objects which will be rendered to the final image.
Thus you should check that your system appears correctly drawn in the 3D view first, and then enter the camera 
view mode to set the viewpoint of your final 3D render.
\par{}
To move the camera, it is recommended that you enter the camera view mode (\enterCamera{}), and then use the fly mode (\flyMode{}) to position the camera using your mouse.
The enter key freezes the camera when you find the viewpoint that you want.
For larger systems, the z-clipping (which hides objects from the view deemed too far from the camera) will 
eliminate parts of your system.
If this happens, select the camera, and then its object Data panel.
Increase the value of 'End' in the 'Clipping' section until the part of your scene you want returns to the camera view.
Pressing the \renderKey{} key will invoke the Blender Render engine and produce a rendered image of your scene from the chosen viewpoint.
The \leaveRender{} key leaves the render mode if you want to make further changes.

The camera can be moved in other ways once in camera view mode. Please see link\sidenote{\navLink{}} for more details.

\section{Summary}
The above information constitutes the minimum required to produce a QCT image from a calculation output file. For further detail on the theoretical background and function of RhoRix, please consult the Manual. For a detailed description of the implementation please consult the paper describing this work\sidenote{\programCitation{}}.
\par{}
It is hoped that the program website\sidenote{\programWebsite{}} will become a repository of completed images and example .blend files in order to inform and inspire future work. Please consider sharing your final Blender save file and rendered image with us, as well as the location of any publications that use this software. Please give credit by citing both the \programName{} paper and the program (as \fullName{}, \programWebsite{}) when \programName{} is used to generate images for publications.

\end{document}

